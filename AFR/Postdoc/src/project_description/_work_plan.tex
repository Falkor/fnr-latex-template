% =============================================================================
% File:  _work_plan.tex --
% Authors: Sebastien Varrette <Sebastien.Varrette@uni.lu>
% 
% Copyright (c) 2013 Sebastien Varrette <Sebastien.Varrette@uni.lu>
% .             http://varrette.gforge.uni.lu
% 
% =============================================================================

\emph{Explanation: Describe the different \textbf{stages of activities} (i.e. work
  packages, milestones) with the corresponding timetable (when?), location
  (where?) and collaborating research groups (with whom?), preferably in
  spreadsheet format.
  The work plan must typically be limited to a period of 2 years.
}



Example of Gantt Chart using \texttt{gantt.sty}\footnote{Credits: Benoit
  Bertholon \url{Benoit.Bertholon@uni.lu}.}

\begin{figure}[H]
    \hspace{-5em}
    \begin{gantt}{7}{12}
        \begin{ganttitle}
            \titleelement{Jan}{1}
            \titleelement{Feb}{1}
            \titleelement{Mar}{1}
            \titleelement{Apr}{1}
            \titleelement{May}{1}
            \titleelement{Jun}{1}
            \titleelement{Jul}{1}
            \titleelement{Aug}{1}
            \titleelement{Sep}{1}
            \titleelement{Oct}{1}
            \titleelement{Nov}{1}
            \titleelement{Dec}{1}
        \end{ganttitle}
        \ganttbar{implementation, xp}{0}{7}
        \ganttbar{article writing}{0}{8}
        \ganttbar{conference}{3}{3}

        % \ganttbarcon{a consecutive task}{2}{4}
        % \ganttbarcon{another consecutive task}{8}{2}
        \ganttbar{literature}{6}{6}
        \ganttbar{design}{7}{5}
        \ganttbar{implementation}{8}{4}
    \end{gantt}
    \caption{Gantt Chart Example over 1 year}
    \label{fig:timeline}

\end{figure}



% ~~~~~~~~~~~~~~~~~~~~~~~~~~~~~~~~~~~~~~~~~~~~~~~~~~~~~~~~~~~~~~~~
% eof
% 
% Local Variables:
% mode: latex
% mode: flyspell
% mode: auto-fill
% fill-column: 80
% End:

% =============================================================================
% _2_research_project.tex --
% 
% Time-stamp: <Mer 2014-03-26 17:02 svarrette>
% =============================================================================

\emph{Remember to define the macro} \verb!\project! \emph{in the \LaTeX\ file
  \texttt{\_\_config.sty}}. 

% ---------------------------------------------------------
\subsection{Research Project}

\begin{center}
    \small
    \begin{tabular}{|p{0.4\textwidth}|p{0.5\textwidth}|}
        \hline
        \textbf{Project Title} {\small \emph{(max 20 words)}}
        & 
        \\\hline
        \textbf{Project Acronym}              & \project\\\hline
        \textbf{Project Start}                & 01/01/2015 \\
        \textbf{Project Duration} (in months) & 36 \\\hline       
        \textbf{CORE Junior Track} (Y/N)      & No\\
        \hline
        Resubmission or Follow Up             & No \\
        \hfill If Yes: Reference              &    \\\hline
     \end{tabular}
\end{center}

% ---------------------------------------------------------
\subsection{Research Priority in FNR Call}

\textbf{Addressed Call Topic: \fbox{IS}}\\

\noindent
\textbf{Accordance with CORE thematic research priorities}
\emph{(max 0.5 page).
Count the number of words in the abstract by running in command-line:
}
\begin{verbatim}
    $> detex _project_accordance.tex | wc -w
\end{verbatim}

%=============================================================================
% File:  _project_accordance.tex --             
% To count number of words in this file:
%
%        detex _project_accordance.tex | wc -w
%
% Accordance with CORE thematic research priorities (max 0.5 page): 
%=============================================================================








% ~~~~~~~~~~~~~~~~~~~~~~~~~~~~~~~~~~~~~~~~~~~~~~~~~~~~~~~~~~~~~~~~
% eof
%
% Local Variables:
% mode: latex
% mode: flyspell
% mode: auto-fill
% fill-column: 80
% End:


% ---------------------------------------------------------
\subsection{Project Summary}

\textbf{Key Words characterising the Research Project:}
\keywords
\\

\noindent
\textbf{Publishable Project Abstract}
\emph{(max. 0.5 pages).
Count the number of words in the abstract by running in command-line:
}
\begin{verbatim}
    $> detex _project_abstract.tex | wc -w
\end{verbatim}

\noindent\textbf{\emph{Abstract}:}
%=============================================================================
% File:  _project_abstract.tex --             
% To count number of words in this file:
%
%        detex _project_abstract.tex | wc -w
%
%=============================================================================


% ~~~~~~~~~~~~~~~~~~~~~~~~~~~~~~~~~~~~~~~~~~~~~~~~~~~~~~~~~~~~~~~~
% eof
%
% Local Variables:
% mode: latex
% mode: flyspell
% mode: auto-fill
% fill-column: 80
% End:


% ---------------------------------------------------------
\subsection{Primary and Secondary Domains of the Research Project}

\begin{center}
    \small
    \begin{tabular}{|l|p{0.6\textwidth}|}
        \hline
        \textbf{Primary Domain:}   & PE6 Computer Science and Informatics\\
        \textbf{Secondary Domains} & PE7 Systems and Communication Engineering\\
        \hline
     \end{tabular}
\end{center}








% ~~~~~~~~~~~~~~~~~~~~~~~~~~~~~~~~~~~~~~~~~~~~~~~~~~~~~~~~~~~~~~~~
% eof
% 
% Local Variables:
% mode: latex
% mode: flyspell
% fill-column: 80
% End:

% =============================================================================
% File:  _project_plan.tex --
% 
% Time-stamp: <Mer 2014-03-26 22:59 svarrette>
% =============================================================================

\emph{Explanation:
  Please indicate roughly the different \textbf{stages of activities} (i.e. work
  packages, milestones) with the corresponding timetable (when?), location
  (where?) and collaborating research groups (with whom?), preferably in
  spreadsheet format. A provisional date for thesis submission and/or final
  thesis examination should be indicated. 	
  The work plan must typically be limited to finalise the PhD within a period of
  3 years. Exceptions can be made for applicants who participate in an official
  PhD programme that is initially planned over a period of 4 years (for example
  PhD schools). 
}

As mentioned in the section~\ref{sec:3_3_methods_approach}, the \project\ project is divided into
XX work packages:
\begin{enumerate}
  \item Project/Configuration Management and Quality Control
  \item ...
  \item Validation on a concrete case studies.
  \item Dissemination of the results
\end{enumerate}

\noindent
The tasks in the different work packages are designed to be iterative and at a
given level, most of them can be done in parallel.
\\
WP1 and WP6 are classical management tasks.\\
WP2, WP3, WP4 will constitute the main advances in terms of scientific contributions.
WP5.
The rest of this section details the work packages mentioned above.




Example of Gantt Chart using \texttt{gantt.sty}\footnote{Credits: Benoit
  Bertholon \url{Benoit.Bertholon@uni.lu}.}

\begin{figure}[H]
    \hspace{-5em}
    \begin{gantt}{7}{12}
        \begin{ganttitle}
            \titleelement{Jan}{1}
            \titleelement{Feb}{1}
            \titleelement{Mar}{1}
            \titleelement{Apr}{1}
            \titleelement{May}{1}
            \titleelement{Jun}{1}
            \titleelement{Jul}{1}
            \titleelement{Aug}{1}
            \titleelement{Sep}{1}
            \titleelement{Oct}{1}
            \titleelement{Nov}{1}
            \titleelement{Dec}{1}
        \end{ganttitle}
        \ganttbar{implementation, xp}{0}{7}
        \ganttbar{article writing}{0}{8}
        \ganttbar{conference}{3}{3}

        % \ganttbarcon{a consecutive task}{2}{4}
        % \ganttbarcon{another consecutive task}{8}{2}
        \ganttbar{literature}{6}{6}
        \ganttbar{design}{7}{5}
        \ganttbar{implementation}{8}{4}
    \end{gantt}
    \caption{Gantt Chart Example over 1 year}
    \label{fig:timeline}

\end{figure}



% ~~~~~~~~~~~~~~~~~~~~~~~~~~~~~~~~~~~~~~~~~~~~~~~~~~~~~~~~~~~~~~~~
% eof
% 
% Local Variables:
% mode: latex
% mode: flyspell
% mode: auto-fill
% fill-column: 80
% End:

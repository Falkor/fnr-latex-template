% =============================================================================
% File: AFR_PostDoc_Full_Proposal.tex -- Main LaTeX document for the project
% proposal
% Author(s): Sebastien Varrette <Sebastien.Varrette@uni.lu>
% 
% Copyright (c) 2013 Sebastien Varrette <Sebastien.Varrette@uni.lu>
% .             http://varrette.gforge.uni.lu
% 
% More information on LaTeX: http://www.latex-project.org/
% LaTeX symbol list:
% http://www.ctan.org/tex-archive/info/symbols/comprehensive/symbols-a4.pdf
% =============================================================================

\documentclass[11pt,twoside,a4paper]{article}
../_style.sty


%%%%%%%%%% Define the project acronym in _style.sty %%%%%%%%


%%%%%%%%%%%%%%%%%%%%%%%%%%%%%%%%%%%%%% 
%%%%%%%%%%%%%%%%%%%%%%%%%%%%%%%%%%%%%% 

\title{\textbf{\Large FNR INTER Proposal for the CRIST-ERA Call} \\[1em]
  {\small Full Proposal Application Form \LaTeX\ template\thanks{by Sebastien
      Varrette {\small \texttt{<Sebastien.Varrette@uni.lu>}}}.\\
    \textbf{Draft version \docversion}, compiled on \isodayandtime}
}
\author{}
\date{}

\begin{document}

\maketitle
{\Large
  \begin{center}
      \hrule
      ~\\
      \emph{Project Acronym}: \project

      \emph{Project Full Title}: \fulltitle\\[1em]
  \end{center}
}
\hrule
~\\[2em]
\textbf{Addressed Call Topic (AMCE\footnote{Adaptive Machines in Complex Environments}  or HDC\footnote{Heterogeneous Distributed Computing} )}: \fbox{HDC}

\emph{Coordinator contact point for the proposal}:

\begin{table}[H]
    \centering\small
    \begin{tabular}{|p{0.48\textwidth}|p{0.5\textwidth}|}
        \hline
        Name:                   & \\\hline
        Institution/Department: & \\\hline
        Address:                & \\\hline
        Country:                & \\\hline
        Phone:                  & \\\hline
        Fax:                    & \\\hline
        E-mail:                 & \\\hline
        \hline
    \end{tabular}
\end{table}

\emph{Partners' people involved in the realisation of the project }

\begin{table}[H]
    \centering\small
    \begin{tabular}{|c|c|p{0.4\textwidth}|p{0.2\textwidth}|p{0.15\textwidth}|}
        \hline
        \rowcolor{lightgray}
        \textbf{\#} & \textbf{Country} & \textbf{Institution} & \textbf{PI} & \textbf{Nat. Funding?}\\\hline
        \hline 
        \textbf{1}  & & & &\\\hline
        \textbf{2}  & Luxembourg & \acf{UL} & Pascal Bouvry & Y \\\hline
        \textbf{3}  & & & &\\\hline
        \textbf{4}  & & & &\\\hline
        \textbf{5}  & & & &\\\hline
        \textbf{6}  & & & &\\\hline
        \hline
    \end{tabular}
\end{table}
\textbf{Note}:

\emph{For each partner, precise Name of the co-Investigators, the name of the
  other personnel participating in the project.} 

%\clearpage
~\\[2em]
\textbf{Duration}: \fbox{36 Months}

\textbf{Summary of the project\footnote{Be precise and concise. This summary will be used to select suited reviewers for the proposal.}:} \emph{(publishable abstract, max. 400 words):}
%=============================================================================
% File:  _project_abstract.tex --             
% To count number of words in this file:
%
%        detex _project_abstract.tex | wc -w
%
%=============================================================================


% ~~~~~~~~~~~~~~~~~~~~~~~~~~~~~~~~~~~~~~~~~~~~~~~~~~~~~~~~~~~~~~~~
% eof
%
% Local Variables:
% mode: latex
% mode: flyspell
% mode: auto-fill
% fill-column: 80
% End:


\textbf{Relevance to the topic addressed in the call\footnote{Be precise and concise. Relevance to the topic addressed in the call is an essential eligibility criterion.}:} \emph{(max. 200 words):}
%=============================================================================
% File:  _project_abstract.tex --             
% To count number of words in this file:
%
%        detex _project_abstract.tex | wc -w
%
%=============================================================================


% ~~~~~~~~~~~~~~~~~~~~~~~~~~~~~~~~~~~~~~~~~~~~~~~~~~~~~~~~~~~~~~~~
% eof
%
% Local Variables:
% mode: latex
% mode: flyspell
% mode: auto-fill
% fill-column: 80
% End:


% --------------------------------------------------------------
\section{Detailed project information}

\textbf{General recommendation}:
\begin{enumerate}
  \item The same font and style should be used for the whole proposal (Arial,
    11pt, single spaced).
  \item Please complete all sections
  \item Please adhere to the given word limits.
  \item For the evaluation criteria, please refer to the call announcement. Your
    proposal should include all details required. 
\end{enumerate}

\Attachment{project_description/project_description.pdf}



%=============================================================================
% File:  _acronyms.tex --  List of used acronyms           
%
% Time-stamp: <Mar 2013-08-20 11:56 svarrette>
%=============================================================================

\begin{acronym}\setlength\itemsep{-0.3em}
    \acro{CC}{Cloud Computing}
    \acro{CSC}{Computer Science and Communications}
    \acro{UL}{University of Luxembourg}
\end{acronym}


%~~~~~~~~~~~~~~~~~~~~~~~~~~~~~~~~~~~~~~~~~~~~~~~~~~~~~~~~~~~~~~~~
% eof
%
% Local Variables:
% mode: latex
% mode: flyspell
% mode: auto-fill
% fill-column: 80
% End:



\end{document}

% ~~~~~~~~~~~~~~~~~~~~~~~~~~~~~~~~~~~~~~~~~~~~~~~~~~~~~~~~~~~~~~~~
% eof
% 
% Local Variables:
% mode: latex
% mode: flyspell
% mode: auto-fill
% fill-column: 80
% End:
